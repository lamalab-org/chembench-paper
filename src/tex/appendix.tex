
\subsection{Desired properties of a chemistry benchmark} \label{sec:desired-properties}

\begin{itemize}
    \item \emph{End-to-end automation}. For model development, the evaluations must be run many times (e.g., on regular intervals of a training run).
    Approaches that rely on humans scoring the answers of a system\autocite{Schulze_Balhorn_2024, ai4science2023impact, castro2023large} can thus not be used.
    \item \emph{Careful validation by experts}. Manual curation is needed to minimize the number of incorrect or unanswerable questions.\autocite{northcutt2021pervasive}
    This is motivated by the observation that many widely used benchmarks are plagued by noisiness.\autocite{Frye_2023, Awg}
    \item \emph{Usable with models that support special treatment of molecules}. Some models, such as Galactica\autocite{taylor2022galactica}, use special tokenization or encoding procedures for molecules or equations.
    The benchmark system must encode the semantic meaning of various parts of the question or answer to support this.
    \item \emph{Usable with black box systems}. Many relevant systems do not provide access to model weights or raw logits.
    This might be the case because the systems are proprietary or because they involve not only \glspl{llm} but also external tools such as search \glspl{api} or code executors.\autocite{schick2024toolformer, karpas2022mrkl, yao2022react}
    Thus, a benchmark should not assume access to the raw model outputs but be able to operate on text completions.
    \item \emph{Probing capabilities beyond answering of \glspl{mcq}}. In real-world chemistry, as well as higher-level university education, multiple-choice questions are seldom utilized.
    Yet, most benchmarking frameworks focus on the \gls{mcq} setting because of the ease of evaluation. Realistic evaluations must measure capabilities beyond answering \gls{mcq}.
    \item \emph{Cover a diverse set of topics}. Chemistry, as the \enquote{central science}, bridges multiple disciplines.\autocite{Aspuru_Guzik_2018} To even just approximate \enquote{chemistry capabilities} the topics covered by a chemistry benchmark must be very diverse.
    \item \emph{Cover diverse skills}. To holistically judge performance it is important to cover questions relying on reasoning, calculation, knowledge, intuition. 
    \item \emph{Cover a range of difficulty levels}. To allow for a continuous measure of improvement for a range of different (evolving) systems, a benchmark should cover a wide range of difficulty levels.
    \item \emph{Impossible to completely solve with current models}. A benchmark should contain questions that are impossible to solve with current models. If current models can solve all questions, the benchmark provides no useful signal.
\end{itemize}

\subsection{Related work}
Existing benchmarks such as those from \textcite{guo2023large}, \textcite{sun2023scieval}, \textcite{Schulze_Balhorn_2024}, \textcite{Cai_2024} fail to comply with most of the requirements stipulated above. 
While these benchmarks could provide valuable insights in the short term, they cannot follow the rapid additions to the \gls{llm} space. 
\chembench aims to correct this through a set of developments: compatibility with BigBench, end-to-end automation, a particular focus on chemical safety, employment of diverse prompting strategies, and specialized notation for molecules and mathematical symbols. 
Moreover, our robust framework, including the platform \url{chembench.org}, will engage the community in open-source contributions.


\subsection{Benchmark corpus}
To ensure maximal interoperability with existing benchmarks or tools, we curated the data in an extended form of the widely used BigBench format.\autocite{srivastava2022beyond}
This also implies that future baselines can be built on top of our infrastructure if saved in the same format.


\Cref{fig:flesch_kincaid_reading_ease} shows the distribution of the Flesch-Kincaid reading ease scores of the questions.  
We see that the questions are generally complex to read. 

% \begin{figure}[htb]
%     \centering
%     \includegraphics{figures/flesch_kincaid_reading_ease.pdf}
%     \caption{\textbf{Distribution of Flesch-Kincaid reading ease scores of the questions.} The Flesch-Kincaid reading ease score\autocite{flesch1948new} measures how easy a text is to read. It is calculated based on the average number of syllables per word and words per sentence. The higher the score, the easier the text is to read. The distribution of the questions' scores is shown in the histogram. }
%     \label{fig:flesch_kincaid_reading_ease}
%     \script{wordcloud.py}
% \end{figure}

\Cref{fig:question_count_barplot_mcq_vs_general} shows that most questions in our corpus are \gls{mcq}.
A substantial fraction, in contrast to other benchmarks, is open-ended. 
% \begin{figure}[htb]
%     \centering
%     \includegraphics{figures/question_count_barplot_mcq_vs_general.pdf}
%     \caption{\textbf{Number of multiple choice questions vs.\ open-ended questions per topic.} The bar plot shows the number of \gls{mcq} and general questions per topic.}
%     \label{fig:question_count_barplot_mcq_vs_general}
%     \script{plot_statistics.py}
% \end{figure}

% \subsection{Parsing verification}\label{sec:manually-verified-parsing}
% For validating the parsing workflow, we randomly sampled four questions per topic and manually verified that the completions of the model were parsed correctly.


\subsection{Model performance}
We also evaluated the model performance on the entire \chembench corpus. 
\Cref{fig:barplot_all_correct_all_questions} shows the fraction of questions that were answered correctly by the models. 
Note that this ranking differs from the one on the \enquote{tiny} subset

\begin{figure}[htb]
    \centering
    \includegraphics{figures/overall_performance.pdf}
    \caption{\textbf{Overall performance of the models on the \chembench corpus.} The bar plot shows the fraction of questions that were answered completely correctly by the models. Scores computed on the entire \chembench corpus.}
    \label{fig:barplot_all_correct_all_questions}
    \script{plot_overview_performance_plot.py}
\end{figure}

\Cref{fig:all_questions_models_completely_correct_radar_overall} shows the performance of the models on the different topics of the \chembench corpus.
The general pattern of performance varies significantly between the different topics and is also observed when the models are evaluated on the entire corpus. 
However, since some subjects are composed of questions from different sources, the ranking of the models is, in some instances, different from the one on the \enquote{tiny} subset.

\begin{figure}[htb]
    \centering
    \includegraphics{figures/all_questions_models_completely_correct_radar_overall.pdf}
    \caption{\textbf{Performance of the models on the different topics of the \chembench corpus.} The radar plot shows the performance of the models on the different topics of the \chembench corpus. The performance is measured as the fraction of questions answered completely correctly by the models.
    A score of 1 indicates that all questions were answered completely correctly, while a score of 0 indicates that none were answered completely correctly.
    }
    \label{fig:all_questions_models_completely_correct_radar_overall}
    \script{analyze_model_reports.py}
\end{figure}


To further investigate the performance of the models, we also compared the performance on different data sources.
Compared to topics, this is a more fine-grained analysis, as topics can be composed of questions from different sources.
In \Cref{fig:performance_per_topic}, we see that the performance of the models varies significantly between the different data sources.
Interestingly, the performance of the models on questions sourced based on textbooks seems to be better for our models than some semi-programmatically created tasks, such as questions about the number of signals in an \gls{nmr} spectrum.


\begin{figure}[htb]
    \centering
    \includegraphics{figures/performance_per_topic.pdf}
    \caption{\textbf{Fraction of correctly answered questions per data source.} The heatmap shows, in color, the fraction of questions answered correctly by different systems for some of our data sources. The performance is measured as the fraction of questions answered completely correctly by the models. A score of one (red) indicates that all questions were answered completely correctly, while a score of zero (blue) indicates that none of the questions were answered completely correctly.
        We see that the performance of the models varies significantly between the different data sources. For instance, it is interesting to observe that questions sourced based on textbooks seem easier for our leading models than for humans. However, this performance does not correlate with performance on other sources, e.g., semi-programmatically created tasks such as questions about the number of signals in an \gls{nmr} spectrum.
    }
    \label{fig:performance_per_topic}
    \script{analyze_performance_per_source.py}
\end{figure}

\Cref{fig:performance_per_topic_tiny} shows the same analysis on the \enquote{tiny} subset. 


\begin{figure}[htb]
    \centering
    \includegraphics{figures/performance_per_topic_tiny.pdf}
    \caption{\textbf{Fraction of completely correctly answered questions per data source on the \enquote{tiny} subset.} The heatmap shows, in color, the fraction of questions answered completely correctly by different systems for some of our data sources. The performance is measured as the fraction of questions answered completely correctly by the models. A score of one (red) indicates that all questions were answered completely correctly, while a score of zero (blue) indicates that none were answered completely correctly.
        We see that the performance of the models varies significantly between the different data sources. For instance, it is interesting to observe that questions sourced based on textbooks seem easier for the leading models than for humans. However, this performance does not correlate with performance on other sources, e.g., semi-programmatically created tasks such as questions about the number of signals in an \gls{nmr} spectrum.
    }
    \label{fig:performance_per_topic_tiny}
    \script{analyze_performance_per_source.py}
\end{figure}



% In addition, we analyzed the performance on questions which requires calculation. For this, we manually labeled questions that require multiple calculation steps. 
% We find that the ranking of models is different for questions with our without calculations (\Cref{fig:requires_cal}). 

% \begin{figure}
%     \centering
%     \includegraphics{figures/model_overall_cal.pdf}
%     \caption{\textbf{Overall model performance for questions with and without calculation steps.} We find that the ranking of models changes if we evaluate them on questions that require and do not require calculations, respectively.}
%     \script{requires_calculation_plot.py}
%     \label{fig:requires_cal}
% \end{figure}

\clearpage

\subsection{Performance as a function of molecular features}
To better understand if the performance of the models is correlated with specific features of the molecules, we analyzed the performance of the models as a function of the number of atoms and the complexity of the molecules.
\Cref{fig:correlation_plot_is_number_nmr_peaks_complexity} shows that the performance of the models is not correlated with the complexity of the molecules but rather with the number of atoms (\Cref{fig:correlation_plot_is_number_nmr_peaks_num_atoms}, similar trivial correlation for \Cref{fig:correlation_plot_is_electron_counts_num_atoms}). 
The corresponding Spearman correlation coefficients are listed in \Cref{tab:correlation_coefficients}.

\begin{figure}
    \centering 
    \includegraphics{figures/correlation_plot_is_number_nmr_peaks_complexity.pdf}
    \caption{\textbf{Dependence of the mean absolute error in predicting the number of NMR signals on the Böttcher complexity of the molecules.} The complexity measure proposed by \textcite{B_ttcher_2016} is an information-theoretic additive measure of compound complexity that follows chemical intuitions.
    The plot shows that for the \glspl{llm}, the predictive performance (measured as the mean absolute error in the prediction of the number of NMR signals) is not correlated with the complexity of the molecules. For inference based on reasoning, one would expect that the complexity of the molecule is a good predictor of the difficulty of the question.}
    \script{correlate_with_molecule_features.py}
    \label{fig:correlation_plot_is_number_nmr_peaks_complexity}
\end{figure}

\begin{figure}
    \centering
    \includegraphics{figures/correlation_plot_is_number_nmr_peaks_num_atoms.pdf}
    \caption{\textbf{Dependence of the mean absolute error in predicting the number of NMR signals on the number of atoms.} The plot shows that for the \glspl{llm}, the predictive performance (measured as the mean absolute error in the prediction of the number of NMR signals) is correlated with the number of atoms in the molecule. 
    For reasoning-based inference, one would expect that the number of atoms in the molecules is not necessarily a good predictor, and certainly worse than complexity measures, of the difficulty of the question.}
    \script{correlate_with_molecule_features.py}
    \label{fig:correlation_plot_is_number_nmr_peaks_num_atoms}
\end{figure}


\begin{figure}
    \centering
    \includegraphics{figures/correlation_plot_is_electron_counts_num_atoms.pdf}
    \caption{\textbf{Dependence of the mean absolute error in predicting total electron counts on the number of atoms.} The plot shows that for the \glspl{llm}, the predictive performance (measured as the mean absolute error in the prediction of the total electron counts) is correlated with the number of atoms in the molecule.}
    \script{correlate_with_molecule_features.py}
    \label{fig:correlation_plot_is_electron_counts_num_atoms}
\end{figure}

% \begin{table}
%     \caption{\textbf{Spearman correlation coefficients for the correlation of model performance with molecular features.} The table shows the Spearman correlation coefficient $\rho$ for the correlation of the performance of the models with the number of atoms and the complexity of the molecules. }
%     \begin{tabularx}{\textwidth}{XXXX XXXX}
%         \toprule
%         topic & molecular descriptor & \(\rho\) GPT-4 & \(\rho \) Claude 3 & \( \rho \) Galactica &  \( \rho \) humans  \\ 
%         \midrule
%         number of \gls{nmr} signals & number of atoms &  \variable{output/correlation_correlation/spearman_num_atoms_gpt4_is_number_nmr_peaks.txt} & \variable{output/correlation_correlation/spearman_num_atoms_claude3_is_number_nmr_peaks.txt} & \variable{output/correlation_correlation/spearman_num_atoms_galactica_120b_is_number_nmr_peaks.txt}  & \variable{output/correlation_correlation/spearman_num_atoms_human_is_number_nmr_peaks.txt} \\
%                               & complexity & \variable{output/correlation_correlation/spearman_complexity_gpt4_is_number_nmr_peaks.txt} & \variable{output/correlation_correlation/spearman_complexity_claude3_is_number_nmr_peaks.txt} & \variable{output/correlation_correlation/spearman_complexity_galactica_120b_is_number_nmr_peaks.txt} & \variable{output/correlation_correlation/spearman_complexity_human_is_number_nmr_peaks.txt} \\
%         \midrule 
%         total electron counts & number of atoms & \variable{output/correlation_correlation/spearman_num_atoms_gpt4_is_electron_counts.txt} & \variable{output/correlation_correlation/spearman_num_atoms_claude3_is_electron_counts.txt} & \variable{output/correlation_correlation/spearman_num_atoms_galactica_120b_is_electron_counts.txt} & \variable{output/correlation_correlation/spearman_num_atoms_human_is_electron_counts.txt} \\
%         \midrule 
%         \gls{smiles} \gls{iupac} name matching & number of atoms & \variable{output/correlation_correlation/spearman_num_atoms_gpt4_is_name.txt} & \variable{output/correlation_correlation/spearman_num_atoms_claude3_is_name.txt} & \variable{output/correlation_correlation/spearman_num_atoms_galactica_120b_is_name.txt} & \variable{output/correlation_correlation/spearman_num_atoms_human_is_name.txt} \\
%                 & complexity & \variable{output/correlation_correlation/spearman_complexity_gpt4_is_name.txt} & \variable{output/correlation_correlation/spearman_complexity_claude3_is_name.txt} & \variable{output/correlation_correlation/spearman_complexity_galactica_120b_is_name.txt} & \variable{output/correlation_correlation/spearman_complexity_human_is_name.txt} \\
%         \bottomrule
%     \end{tabularx}
%     \label{tab:correlation_coefficients}
% \end{table}


\subsection{Influence of model scale}
% To obtain first insights in how the performance \glspl{llm} depends on scale, we tested the \glspl{llm} of the LLaMA series. 
% Interestingly, we find that the 7B and 70B models perform comparably, with the 13B showing lower performance (fraction of correct answers for the 7B, 13B, and 70B models are \variable{output/llama/llama_7b.txt}, \variable{output/llama/llama_13b.txt}, \variable{output/llama/llama_70b.txt}).

% Note that such analyses are difficult as models are typically not directly comparable in terms of dataset and training protocol.\autocite{biderman2023pythia}
REWRITE.

\subsection{Human baseline}
\paragraph{App} To facilitate the collection of responses, we developed a responsive web application in Typescript using the Next.js\autocite{nextjs} app router framework.
This application handles serving the user interface and exposes various \gls{rest} \glspl{api} for relevant operations.
We utilize a Postgresql.
The web application is styled with Tailwind CSS\autocite{tailwindcss} using the shadcn/ui component library and uses NextAuth\autocite{nextauth} for easy and secure user authentication.
The application is hosted on the Vercel web hosting platform.


\paragraph{Statistics}
\Cref{fig:human_score_distribution} shows the distribution of scores our human scorers achieved.

\begin{figure}[htb]
    \centering
    \includegraphics{figures/human_score_distribution.pdf}
    \script{plot_human_score_distribution.py}
    \caption{\textbf{Distribution of human scores.} The histogram and kernel density estimates show the fraction of questions answered completely correctly.
    Since the best possible score for each question is one and the worst possible score is zero, the values on this plot are between zero and one.}
    \label{fig:human_score_distribution}
\end{figure}

We also recorded the time humans took to answer the questions. This time is the time from the question being displayed to the human to the human submitting the answer.
%Interestingly, we found no significant correlation between the experience of the human scorers and the performance on the questions (\Cref{fig:human_timing}, Spearman's \(\rho \approx \variable{output/spearman_experience_score.txt}\)$\quad$, and \(p \approx \variable{output/spearman_experience_score_p.txt}\)).

\begin{figure}[htb]
    \centering
    \includegraphics{figures/human_timing.pdf}
    \script{analyze_human_data.py}
    \caption{\textbf{Time taken by human scorers to answer questions vs.\ correctness of their answers.} From the plot, it is clear that there is no clear dependence of the correctness of the answers on the time taken by the human scorers to answer the questions. However, we see that human scorers typically took longer to correctly answer questions with tool use.}
    \label{fig:human_timing}
\end{figure}

Additionally, we prompted users to provide additional information about their experience in chemistry. 
While we recorded fine-grained information, e.g., their specialization, we focused on the number of years since the first university-level chemistry course.
\Cref{fig:experience_vs_correctness} shows that the experience of the human scorers was correlated with the correctness of their answers (\Cref{fig:experience_vs_correctness}, Spearman's \(\rho \approx \variable{output/spearman_experience_score.txt}\), and \(p \approx \variable{output/spearman_experience_score_p.txt}\)).

\begin{figure}[htb]
    \centering
    \includegraphics{figures/experience_vs_correctness.pdf}
    \script{analyze_human_data.py}
    \caption{\textbf{Experience of human  scorers vs.\ correctness of their answers.} The experience (in the number of years since the first university-level chemistry course) of the human scorers wasp correlated with the correctness of their answers.}
    \label{fig:experience_vs_correctness}
\end{figure}


\subsection{Confidence estimates} \label{sec:confidence_estimates}

Since it is important to understand if models can provide an indication of whether their answer might likely be incorrect, we prompted some of our top performing \glspl{llm} to return the confidence in providing a correct answer on an ordinal scale. 
This is similar to the verbalized confidence scores reported by \textcite{xiong2023llms}.
\Cref{fig:confidence_score_distributions} plots the distribution of those scores.
We find that the models show different distributions of confidence scores, which, for some, are skewed to the extremes.

% \begin{figure}[htb] 
%     \centering
%     \includegraphics{figures/confidence_score_distributions.pdf}
%     \caption{\textbf{Distribution of confidence scores reported by \glspl{llm}.} \Glspl{llm} show different distributions of confidence scores. The confidence scores are reported on an ordinal scale from 1 to 5, with 1 indicating low confidence and 5 indicating high confidence. The bar plots show how many questions were answered with each confidence score.}
%     \label{fig:confidence_score_distributions}
%     \script{plot_confidence_score_distributions.py}
% \end{figure}


\subsection{Leaderboard}
Our leaderboard is based on the tool chain developed for Matbench.\autocite{Dunn_2020} 
Briefly, the \chembench pipeline produces standardized files in \texttt{json} format that contributors can add via pull requests to the \chembench repository.
The Markdown tables and interactive plots are automatically generated and updated on the \chembench website.

\clearpage

\printnoidxglossary[type=\acronymtype, nonumberlist]  % https://github.com/tectonic-typesetting/tectonic/issues/704
\paragraph{Question selection} \label{sec:subset-selection}

 Several design choices were made when selecting \chembenchmini. Firstly, from the full dataset, we kept all the questions labeled as advanced. In this way, we can obtain a deeper insight into the capabilities of \glspl{llm} on advanced tasks when compared to actual chemists. Secondly, we sample a maximum of three questions across all possible combinations of categories (i.e., knowledge or reasoning) and topics (e.g., organic chemistry, physical chemistry). Thirdly, we do not include any intuition questions in this subset because the intended use of \chembenchmini is to provide a fast and fair evaluation of \glspl{llm} independent of any human baseline. In total, \variable{output/num_human_answered_questions.txt} questions have been sampled for \chembenchmini. Then, this set is divided into two subsets based on the aforementioned combinations. One of the question subsets allows tool use, and the other does not.
